\chapter{二、开题报告}

\section{问题提出的背景}

\subsection{背景介绍}

\subsubsection{监督学习与无监督学习}



\subsubsection{卷积神经网络}



\subsection{本研究的意义和目的}



\section{论文的主要内容和技术路线}


\subsection{主要研究内容}



\subsection{技术路线}

\subsubsection{卷积网络}


\subsection{可行性分析}

\subsubsection{数据收集的可行性}


\subsubsection{卷积网络的可行性}


\subsubsection{多尺度的可行性}


\section{研究计划进度安排及预期目标}

\subsection{进度安排}

本项目的整体计划如下表所示:

\begin{table}[!htbp]
\centering
\begin{tabular}{|l|l|}
\hline
日期 & 主要工作 \\ \hline
2017-07-01 \textasciitilde\ 2017-07-26 & 确定题目,阅读计算机视觉相关论文,学习统计学相关的概念 \\ \hline
\end{tabular}
\caption{项目进度计划}
\label{table:schedule}
\end{table}

\subsection{预期目标}

\begin{itemize}
	\item{生成各种类别的图片,包括人脸、森林、火山、旅店房间等}
	\item{将模型的中间步骤提取出图片特征后在分类问题上进行实验}
	\item{将此生成式模型应用于图像修复}
	\item{所有程序源文件}
	\item{本设计最终成果报告}
\end{itemize}



\section{参考文献}


\begin{itemize}
	\item [{[}1{]}] Y. LeCun, S. Chopra, R. Hadsell, M. Ranzato, and F. J. Huang. A tutorial on energy-based learning. 2006.
	\item [{[}2{]}] G. E. Hinton. Training products of experts by minimizing contrastive divergence. Neural Computation, 14(8):1771– 1800, 2002.
	\item [{[}3{]}] G. E. Hinton, S. Osindero, M. Welling, and Y.-W. Teh. Unsupervised discovery of nonlinear structure using contrastive backpropagation. Cognitive Science, 30(4):725–731, 2006.
	\item [{[}4{]}] G. E. Hinton, S. Osindero, and Y.-W. Teh. A fast learning algorithm for deep belief nets. Neural Computation, 18:1527– 1554, 2006.
	\item [{[}5{]}] R. Salakhutdinov and G. E. Hinton. Deep boltzmann machines. In AISTATS, 2009.
	\item [{[}6{]}] H. Lee, R. Grosse, R. Ranganath, and A. Y. Ng. Convolutional deep belief networks for scalable unsupervised learning of hierarchical representations. In ICML, pages 609–616. ACM, 2009.
	\item [{[}7{]}] J. Ngiam, Z. Chen, P. W. Koh, and A. Y. Ng. Learning deep energy models. In International Conference on Machine Learning, 2011.
	\item [{[}8{]}] Y. Lu, S.-C. Zhu, and Y. N. Wu. Learning FRAME models using CNN filters. In Thirtieth AAAI Conference on Artificial Intelligence, 2016.
	\item [{[}9{]}] J. Xie, Y. Lu, S.-C. Zhu, and Y. N. Wu. A theory of generative convnet. In ICML, 2016.
	\item [{[}10{]}] J. Xie, S.-C. Zhu, and Y. N. Wu. Synthesizing dynamic patterns by spatial-temporal generative convnet. In CVPR, 2017.
	\item [{[}11{]}] L. Jin, J. Lazarow, and Z. Tu. Introspective learning for discriminative classification. In Advances in Neural Information Processing Systems, 2017.
	\item [{[}12{]}] Nasrabadi, Nasser M. "Pattern recognition and machine learning." Journal of electronic imaging 16, no. 4 (2007): 049901.
	\item [{[}13{]}] Deng, Jia, Wei Dong, Richard Socher, Li-Jia Li, Kai Li, and Li Fei-Fei. "Imagenet: A large-scale hierarchical image database." In Computer Vision and Pattern Recognition, 2009. CVPR 2009. IEEE Conference on, pp. 248-255. IEEE, 2009.
	\item [{[}14{]}] Hastie, Trevor, Robert Tibshirani, and Jerome Friedman. "Unsupervised learning." In The elements of statistical learning, pp. 485-585. Springer, New York, NY, 2009.
	\item [{[}15{]}] Jaakkola, Tommi, and David Haussler. "Exploiting generative models in discriminative classifiers." In Advances in neural information processing systems, pp. 487-493. 1999.
	\item [{[}16{]}] Eddy, Sean R. "Hidden markov models." Current opinion in structural biology 6, no. 3 (1996): 361-365.
	\item [{[}17{]}] Goodfellow, Ian, Jean Pouget-Abadie, Mehdi Mirza, Bing Xu, David Warde-Farley, Sherjil Ozair, Aaron Courville, and Yoshua Bengio. "Generative adversarial nets." In Advances in neural information processing systems, pp. 2672-2680. 2014.
	\item [{[}18{]}] Y. LeCun, L. Bottou, Y. Bengio, and P. Haffner. Gradientbased learning applied to document recognition. Proceedings of the IEEE, 86(11):2278–2324, 1998.
	\item [{[}19{]}] A. Krizhevsky, I. Sutskever, and G. E. Hinton. Imagenet classification with deep convolutional neural networks. In NIPS, pages 1097–1105, 2012.
	\item [{[}20{]}] T. Tieleman. Training restricted boltzmann machines using approximations to the likelihood gradient. In Proceedings of the 25th international conference on Machine learning, pages 1064–1071. ACM, 2008.
	\item [{[}21{]}] J. J. Hopfield. Neural networks and physical systems with emergent collective computational abilities. Proceedings of the national academy of sciences, 79(8):2554–2558, 1982. 3, 8
	\item [{[}22{]}] J. Dai, Y. Lu, and Y. N. Wu. Generative modeling of convolutional neural networks. In ICLR, 2015.
	\item [{[}23{]}] S.-C. Zhu and D. Mumford. Grade: Gibbs reaction and diffusion equations. In ICCV, pages 847–854, 1998.
	\item [{[}24{]}] M. Girolami and B. Calderhead. Riemann manifold langevin and hamiltonian monte carlo methods. Journal of the Royal Statistical Society: Series B (Statistical Methodology), 73(2):123–214, 2011.
	\item [{[}25{]}] R. M. Neal. Mcmc using hamiltonian dynamics. Handbook of Markov Chain Monte Carlo, 2, 2011.
	\item [{[}26{]}] J. Dai, Y. Lu, and Y. N. Wu. Generative modeling of convolutional neural networks. In ICLR, 2015.
	\item [{[}27{]}] S. Kirkpatrick, C. D. Gelatt, M. P. Vecchi, et al. Optimization by simulated annealing. science, 220(4598):671–680, 1983.
	\item [{[}28{]}] Goodfellow, Ian. "NIPS 2016 tutorial: Generative adversarial networks." arXiv preprint arXiv:1701.00160 (2016).
	\item [{[}29{]}] Radford, Alec, Luke Metz, and Soumith Chintala. "Unsupervised representation learning with deep convolutional generative adversarial networks." arXiv preprint arXiv:1511.06434 (2015).
	\item [{[}30{]}] Gehring, Jonas, Michael Auli, David Grangier, Denis Yarats, and Yann N. Dauphin. "Convolutional sequence to sequence learning." arXiv preprint arXiv:1705.03122 (2017).
	\item [{[}31{]}] Chen, Qiming, and Ren Wu. "CNN Is All You Need." arXiv preprint arXiv:1712.09662 (2017).
	\item [{[}32{]}] Xie, Jianwen, Yang Lu, Song-Chun Zhu, and Ying Nian Wu. "Cooperative training of descriptor and generator networks." arXiv preprint arXiv:1609.09408 (2016).
	\item [{[}33{]}] Welling, Max, and Yee W. Teh. "Bayesian learning via stochastic gradient Langevin dynamics." In Proceedings of the 28th International Conference on Machine Learning (ICML-11), pp. 681-688. 2011.
\end{itemize}

% 按文章长度需要启用
%\ifthenelse{\equal{\zjuside}{T}}{\newpage\mbox{}\thispagestyle{empty}}{}

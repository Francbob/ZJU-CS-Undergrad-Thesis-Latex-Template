\chapternonum{浙江大学本科生毕业论文任务书}

{
	\setlength{\parindent}{0em}
	\renewcommand{\baselinestretch}{2}
	\fangsong\xiaosi\bfseries
	
	一、 \; 题目: \; 基于对比散度学习多尺度生成式卷积网络
	\vspace{2em}

	二、 \; 指导教师对毕业论文的进度安排及任务要求:
	\vspace{2em}
	
}	

{
	\fangsong\xiaosi
	在进度安排方面,建议从七月份开始就确定好题目,并且阅读计算机视觉相关论文,学习统计学相关的概念。随后可以阅读生成式模型的相关论文,对各种生成式模型形成了解,然后针对之前的模型不足的地方思考解决办法,并且调研各种公开数据集。九月份开始可以阅读论文了解评价生成式模型的各种指标以及目前的最高水平,随后十月份可以熟悉常用的一些深度学习框架并复现各种基线算法及其他论文。十一月份建议开始熟悉编程将要用的框架,实现自己对模型的改进,随后改进模型结构,优化模型参数,训练深度模型,接着在年底跑各种实验并且记录实验结果,用相同实验任务测试其他模型。
}

	\vspace{2cm}

{
	\setlength{\parindent}{0em}
	\fangsong\xiaosi\bfseries
	
	起讫日期 ~~ 2017 年  7 月 1 日 至 2018 年 5 月 1 日
	
	\begin{flushright}
		指导教师(签名) \; \underline{\hspace{6em}} ~~~~ 职称 \; \underline{\hspace{3em}}\\
		年 \qquad 月 \qquad 日
	\end{flushright}
}


{
	\setlength{\parindent}{0em}
	\renewcommand{\baselinestretch}{2}
	\fangsong\xiaosi\bfseries
	
	三、 \; 系或研究所审核意见:
	
	\vspace{2cm}
}

{
	\fangsong\xiaosi\bfseries
	
	\begin{flushright}
	负责人(签名) \; \underline{\hspace{6em}} \\
	年 \qquad 月 \qquad 日
	\end{flushright}
}

\newpage

\chapternonum{浙江大学本科生毕业论文考核表}

{
	\setlength{\parindent}{0em}
	\renewcommand{\baselinestretch}{2}
	\fangsong\xiaosi\bfseries
	
	一、 \; 指导教师对毕业论文的评语:
	\vspace{8em}
	
	\begin{flushright}
		指导教师(签名) \; \underline{\hspace{6em}} \\
		年 \qquad 月 \qquad 日
	\end{flushright}

	\vspace{1em}

	二、 \; 答辩小组对毕业论文(设计)的答辩评语及总评成绩:
	\vspace{8em}
	
	\begin{table}[H]
		\centering \bfseries \wuhao
		\begin{tabularx}{\textwidth}{|>{\fangsong}c
				|>{\fangsong}X<{\centering}
				|>{\fangsong}X<{\centering}
				|>{\fangsong}X<{\centering}
				|>{\fangsong}X<{\centering}
				|>{\fangsong}c|}
			\hline
			\makecell{成绩\\比例} & \makecell{文献综述\\(10\%)}& \makecell{开题报告\\(15\%)} & \makecell{外文翻译\\(5\%)} & \makecell{毕业论文\\质量及答辩\\(70\%)} & \makecell{总评成绩} \\
			\hline
			分值 &  &  &  &  &  \\
			~ & ~ & ~ & ~ & ~ & ~ \\
			\hline
		\end{tabularx}
	\end{table}
	
	\begin{flushright}
	答辩小组负责人(签名) \; \underline{\hspace{6em}} \\
	年 \qquad 月 \qquad 日
	\end{flushright}
	
}	
